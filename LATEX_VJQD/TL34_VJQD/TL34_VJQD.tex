\documentclass[12pt]{article}
\usepackage[utf8]{inputenc}
\usepackage[spanish]{babel}
\usepackage[top=1cm, bottom=2.5cm, right=3cm,  left=3.5cm]{geometry}
\usepackage{amsthm}
\swapnumbers
\title{\textbf{Y colorín, corolario, este tema se ha acabado}}
\author{Verónica Jackeline Quiros Díaz  TL34\_VJQD}
\date{July 2020}

\begin{document}
\maketitle

\section{Ejemplos de Enunciados}
\theoremstyle{definition}
\newtheorem{def1}
{Definición}[section]
\begin{def1}
\label{def1}
    Sea $u$ un entero positivo. Un código $C$ se dice \textit{detector de u errores} si, siempre que una palabra código incurre en al menos un error  y a los más $u$ errores, la palabra resultante no es una palabra código. Un código C se dice \textit{detector de exactamente u errores} si es detector de $(u+1)$ errores.
\end{def1}

\theoremstyle{remark}
\newtheorem{ej1}
[def1]{Ejemplo}
\begin{ej1}
    Sea el código binario $C=\lbrace00000,00111,11222\rbrace$. Analicemos los cambios necesarios en las coordenadas de cada palabra código de manera que podamos obtener alguna otra palabra código existente en C. 
\end{ej1}

\theoremstyle{plain}
\newtheorem{th1}
[def1]{Teorema}
\begin{th1}
\label{th1}
    Un código C es \emph{detector de u errores} si y solo si $d(C)\geq u+1$, es decir, un código con distancia d es un código corrector de exactamente $(d-1)$ errores.
\end{th1}
\proof La prueba se sigue por definición.
\end{document}
