%%%Composición de formulas matematicas 
\documentclass[12pt]{article}
\usepackage[utf8]{inputenc}
\usepackage[spanish]{babel}
%%%Paqueterias para conceptos matematicos
\usepackage{amsmath}
\usepackage{amssymb}
%bordes de la pagina {Margenes}
\setlength{\topmargin}{0mm}
%%%Preambulo
\title{TL32\_VJQD}
\author{Veronica Quiros}
\date{June 2020}
%Cuerpo
\begin{document}
\maketitle
%% Ejemplos de formulas con Latex
 \section{Entornos Matem\'aticos}
 \subsection{Expresiones}
 \begin{enumerate}
     \item 
     \begin{align*}
        h_{wH}(\delta)&= \min\limits_{z\in(0,1]} \log_q \frac{f_{wH} (z)}{z^\delta}\\
        & =\min\limits_{z\in(0,1]} (\log_q {f_{wH} (z)}-\log_{q} {z^\delta})\\
        &=\min\limits_{z\in(0,1]}(\log_{q} (1+(q-1)z) -\delta\log_{q} z)\\
        &=\log_{q} \left(1+(q-1)\frac{\delta}{(q-1)(1-\delta)}\right)-\delta\log_{q} \left(\frac{\delta}{(q-1)(1-\delta)}\right)\\
        &=\log_{q} \left(\frac{1}{(1-\delta)}\right) - \delta\log_{q} \delta +\delta\log_{q} (q-1) +\delta\log_{q} (q-1)\\ 
        &=\log_{q} \frac{1}{\delta} +(1-\delta)\log_{q} \frac{1}{1-\delta}+\delta\log_{q} (q-1)\\
     \end{align*}
     \item
     \begin{align*}
         ab&=[x_{1},x_{2}]qx_{2}[x_{1},x_{2}][x_{1},x_{2}]x_{1}+q^{-1}qx_{2}[x_{1},x_{2}][[x_{1},x_{2}]+ q^{-1}x_{2}x_{1}][x_{1},x_{2}]x_{1}\\
         &=[x_{1},x_{2}]qx_{2}[x_{1},x_{2}][x_{1},x_{2}]x_{1}+x_{2}[x_{1},x_{2}][x_{1},x_{2}][x_{1},x_{2}]x_{1}\\
         & \quad+x_{2}[x_{1},x_{2}]q^{-1} x_{2}x_{1}[x_{1},x_{2}]x_{1}\\
     \end{align*}
     \item
     \begin{align}
         [x_{i},x_{j}]&=0, \quad si \, \left| i-j\right| > 1;\\
         [[x_{i},x_{i+1}],x_{i+1} ]&=0, \quad si \; 1 \leq i < n;\\
         [x_{i},[x_{i},x_{i+1}] ]&=0, \quad si \; 1 \leq i < n.
     \end{align}
     \item
        \begin{align*}
         [x_{i},x_{j}]&=0, \quad si \, \left| i-j\right| > 1;\\
         [[x_{i},x_{i+1}],x_{i+1} ]&=0, \quad si \; 1 \leq i < n;
         \end{align*}
         \begin{align}
         [x_{i},[x_{i},x_{i+1}] ]&=0, \quad si \; 1 \leq i < n.
         \end{align}
     \item
     \begin{align*}
         e^{i\theta_{1}} e^{i\theta_{2}} &= (\cos \theta_{1} + i\sin \theta_{1})(\cos \theta_{2} + i\sin \theta_{2})\\
         &=(\cos \theta_{1} \cos \theta_{2} - \sin \theta_{1} \sin \theta_{2})+ i(\cos \theta_{1} \sin \theta_{2} + \sin \theta_{1} \cos \theta_{2})\\
         &=\cos (\theta_{1} + \theta_{2})+ \sin (\theta_{1} + \theta_{2})\\
         &=e^{i(\theta_{1}+\theta_{2})}
     \end{align*}
 \end{enumerate}

\end{document}
