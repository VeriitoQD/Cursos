\documentclass{article}
\usepackage[utf8]{inputenc}
\usepackage[spanish,es-tabla]{babel}
\usepackage[top=1.5cm,bottom=2cm,right=2cm,left=2.5cm]{geometry}
\usepackage{float}
\usepackage{multirow}
\usepackage{multicol}
\usepackage{array}
\usepackage{longtable} 
\usepackage{ragged2e} 
\usepackage{lscape}
\title{\textbf{Tablas con \LaTeX}}
\author{TL42\_VJQD}
\date{July 2020}
\begin{document}
\maketitle
\justify A continuación, se presentan algunos ejemplos de tablas creadas con \LaTeX. Los entornos {longtable} y {landscape} se distribuyen con MikTeX, por lo que no es necesario descargarlos de ningún repositorio.\\\\

\centering\listoftables
\newpage
%%%--------------a--------------
\begin{table}[H]
\centering
\begin{tabular}{p{2.5cm}p{8cm}}
\hline
\hline
\textbf{Término} & \textbf{Descripción}\\
\hline
\LaTeX perto & Usuario avanzado de \LaTeXe\ capacitado para la obtención y empleo de paquetes, herramientas y/o aplicaciones desarrolladas para lograr sus propósitos al realizar documentos con \LaTeX.\\
\hline
\LaTeX plorador & Usuario principiante de \LaTeXe\ cuya ocupación principal consiste en la recopilación de fuentes de información sobre el uso de \LaTeX.\\
\hline
\LaTeX iliado & Usuario de procesadores de texto convencionales que insiste en que no necesita \LaTeX\ para nada, o bien, en que éste no le es de utilidad.\\
\hline
\LaTeX cluido & Usuario de procesadores de texto convencionales que desconoce la existencia de \LaTeX.\\
\hline
\hline
\end{tabular}
\caption{Tabla con especificación de ancho de columna}
\label{tab:1}
\end{table}
%------------------b--------------------------
\begin{table}[H]
\centering
\begin{tabular}{p{2.5cm}p{8cm}}
\hline
\hline
\multicolumn{1}{c}{\textbf{Término}} & \multicolumn{1}{c}{\textbf{Descripción}}\\
\hline
\LaTeX perto & Usuario avanzado de \LaTeXe\ capacitado para la obtención y empleo de paquetes, herramientas y/o aplicaciones desarrolladas para lograr sus propósitos al realizar documentos con \LaTeX.\\
\hline
\LaTeX plorador & Usuario principiante de \LaTeXe\ cuya ocupación principal consiste en la recopilación de fuentes de información sobre el uso de \LaTeX.\\
\hline
\LaTeX iliado & Usuario de procesadores de texto convencionales que insiste en que no necesita \LaTeX\ para nada, o bien, en que éste no le es de utilidad.\\
\hline
\LaTeX cluido & Usuario de procesadores de texto convencionales que desconoce la existencia de \LaTeX.\\
\hline
\hline
\end{tabular}
\caption{Tabla con encabezados centrados}
\label{tab:2}
\end{table}
%--------------------c------------------------
\begin{table}[H]
\footnotesize
\centering
\begin{tabular}{p{2.5cm}p{8cm}}
\hline
\hline
\textbf{Término} & \textbf{Descripción}\\
\hline
\LaTeX perto & Usuario avanzado de \LaTeXe\ capacitado para la obtención y empleo de paquetes, herramientas y/o aplicaciones desarrolladas para lograr sus propósitos al realizar documentos con \LaTeX.\\
\hline
\LaTeX plorador & Usuario principiante de \LaTeXe\ cuya ocupación principal consiste en la recopilación de fuentes de información sobre el uso de \LaTeX.\\
\hline
\LaTeX iliado & Usuario de procesadores de texto convencionales que insiste en que no necesita \LaTeX\ para nada, o bien, en que éste no le es de utilidad.\\
\hline
\LaTeX cluido & Usuario de procesadores de texto convencionales que desconoce la existencia de \LaTeX.\\
\hline
\hline
\end{tabular}
\caption{Tabla con tamaño de fuente igual al de los pies de página}
\label{tab:3}
\end{table}
%%------------------------d------------------------
\begin{table}[H]
\centering
\begin{tabular}{p{2.5cm}p{8cm}}
\hline
\hline
\textbf{Término} & \textbf{Descripción}\\
\hline
\LaTeX perto & Usuario avanzado de \LaTeXe\ capacitado para la obtención y empleo de paquetes, herramientas y/o aplicaciones desarrolladas para lograr sus propósitos al realizar documentos con \LaTeX.\\
\hline
\LaTeX perto & Usuario avanzado de \LaTeXe\ capacitado para la obtención y empleo de paquetes, herramientas y/o aplicaciones desarrolladas para lograr sus propósitos al realizar documentos con \LaTeX.\\
\hline
\LaTeX perto & Usuario avanzado de \LaTeXe\ capacitado para la obtención y empleo de paquetes, herramientas y/o aplicaciones desarrolladas para lograr sus propósitos al realizar documentos con \LaTeX.\\
\hline
\LaTeX plorador & Usuario principiante de \LaTeXe\ cuya ocupación principal consiste en la recopilación de fuentes de información sobre el uso de \LaTeX.\\
\hline
\LaTeX plorador & Usuario principiante de \LaTeXe\ cuya ocupación principal consiste en la recopilación de fuentes de información sobre el uso de \LaTeX.\\
\hline
\LaTeX plorador & Usuario principiante de \LaTeXe\ cuya ocupación principal consiste en la recopilación de fuentes de información sobre el uso de \LaTeX.\\
\hline
\LaTeX iliado & Usuario de procesadores de texto convencionales que insiste en que no necesita \LaTeX\ para nada, o bien, en que éste no le es de utilidad.\\
\hline
\LaTeX iliado & Usuario de procesadores de texto convencionales que insiste en que no necesita \LaTeX\ para nada, o bien, en que éste no le es de utilidad.\\
\hline
\LaTeX iliado & Usuario de procesadores de texto convencionales que insiste en que no necesita \LaTeX\ para nada, o bien, en que éste no le es de utilidad.\\
\hline
\LaTeX cluido & Usuario de procesadores de texto convencionales que desconoce la existencia de \LaTeX.\\
\hline
\LaTeX cluido & Usuario de procesadores de texto convencionales que desconoce la existencia de \LaTeX.\\
\hline
\LaTeX cluido & Usuario de procesadores de texto convencionales que desconoce la existencia de \LaTeX.\\
\hline
\hline
\end{tabular}
\caption{Tabla que ocupa más de una página}
\label{tab:4}
\end{table}
%%------------------------f---------------------
\begin{landscape}
\begin{table}[H]
\centering
\begin{tabular}{p{2.5cm}p{8cm}}
\hline
\hline
\textbf{Término} & \textbf{Descripción}\\
\hline
\LaTeX perto & Usuario avanzado de \LaTeXe\ capacitado para la obtención y empleo de paquetes, herramientas y/o aplicaciones desarrolladas para lograr sus propósitos al realizar documentos con \LaTeX.\\
\hline
\LaTeX plorador & Usuario principiante de \LaTeXe\ cuya ocupación principal consiste en la recopilación de fuentes de información sobre el uso de \LaTeX.\\
\hline
\LaTeX iliado & Usuario de procesadores de texto convencionales que insiste en que no necesita \LaTeX\ para nada, o bien, en que éste no le es de utilidad.\\
\hline
\LaTeX cluido & Usuario de procesadores de texto convencionales que desconoce la existencia de \LaTeX.\\
\hline\hline
\end{tabular}
\caption{Tabla ``acostada''}
\label{tab:5}
\end{table}
\end{landscape}

\end{document}
