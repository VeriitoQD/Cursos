\documentclass[12pt]{article}
\usepackage[utf8]{inputenc}
\usepackage[spanish]{babel}
\usepackage[top=3cm, bottom=3cm,right=3cm,left=3cm]{geometry}
\title{\textbf{Bibliografía Artesanal}}
\author{TL51\_VJQD}
\date{July 2020}
\begin{document}
\maketitle
\begin{enumerate}
    \item Existen diversas estrategias para minimizar el error de redondeo (véase \cite{Akai}).
        \begin{thebibliography}{XXX99}
            \bibitem{Akai}
            Akai, Terrence J. \emph{Métodos numéricos}. Limusa, México, 2002.
        \end{thebibliography}
    \item Existen diversas estrategias para minimizar el error de redondeo (véase \cite{Akai2}).
        \begin{thebibliography}{XXX99}
            \bibitem[A202]{Akai2}
            Akai, Terrence J. \emph{Métodos numéricos}. Limusa, México, 2002. 
        \end{thebibliography}
    \item Existen diversas estrategias para minimizar el error de redondeo \footnote{
            Akai, Terrence J. \emph{Métodos numéricos}. Limusa, México, 2002.}.
    \item Existen diversas estrategias para minimizar el error de redondeo (véase \cite{Akai4})
        \begin{thebibliography}{99}
            \bibitem{Akai4}
            Akai, T J. \emph{Métodos numéricos}. Limusa, México, 2002. 
        \end{thebibliography}
\end{enumerate}
\end{document}
