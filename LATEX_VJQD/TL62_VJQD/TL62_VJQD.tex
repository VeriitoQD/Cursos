%% "Plantilla para elaborar presentaciones con LaTeX Beamer”%%
\documentclass[11pt]{beamer}
\usepackage[utf8]{inputenc}
\usepackage[spanish]{babel}
\usepackage{amsmath}
\usepackage{soul}
\usepackage{color}
\usepackage{amsfonts}
\usepackage{amssymb}
\usepackage{graphicx}
\usepackage{lipsum}
\usepackage{ragged2e}
\usepackage{hyperref}
\usepackage{float}
\usepackage{url} 
\usepackage{verbatim}
\usetheme{Warsaw}

\author[TL62\_VJQD]{Verónica Jackeline Quiros Díaz}
\title[Clase Beamer]{Una plantilla para usar Beamer}
\date{20 de Agosto de 2020} 
\subtitle{Presentación básica en \LaTeX}


\begin{document}
\maketitle

    \begin{frame}{Índice}
		\tableofcontents
	\end{frame}
    \section{Programación}
		\begin{frame}{Programación}
		Es una serie de instrucciones que se sigan para realizar una operación, escritas mediante un lenguaje de programación.
		\end{frame}
    \subsection{Pseudocódigo}
            \begin{frame}{Pseudocódigo}
            \centering
			Es la manera de expresar el procedimiento o bien, la solución de un problema, pero escrito en lenguaje natural$($ utilizamos para comunicarnos$).$
			\end{frame}
	\subsection{Algoritmo}
            \begin{frame}{Algoritmo}
			\justifying
			Es la manera real de plasmar el procedimiento o bien, la solución de un problema. Se respalda de técnicas y fundamentos.
			\end{frame}
    \section{Lenguajes}
		\begin{frame}{Lenguajes}
			\justifying
			Es la manera de entablar comunicación con la máquina, ya que funje como traductor o intermediario: transforma el lenguaje natural al de máquina, para que siga las instrucciones y resuelva un problema.
		\end{frame}
	\subsection{Código}	
	    \begin{frame}{Código}
			\justifying
			Es el conjunto de lineas de texto, que indica los pasos que se deben seguir para elaborar un programa.
		\end{frame}
	\subsection{Variedad}	
	    \begin{frame}{Variedad}
			\justifying
			Existe una gran lista de lenguajes de progración. Cada uno de ellos se dedica a programar ciertas partes del software. Se encuentran tanto para frontend como para la parte del backend, entre otros.
			Ejemplos:
			\begin{itemize}
			    \item C,C++
			    \item Java
			    \item Javascript
			    \item Python
			    \item PHP,HTML,CSS
			    \item etc.
			\end{itemize}
		\end{frame}
	
    \section{Estructuras de Datos}
		\begin{frame}{Estructuras de Datos}
			\centering
			\begin{columns}
                \column{2in}
                Como su nombre lo indica, son estruturas o bien, metodos(operaciones) que permiten reducir carga de trabajo, tanto en complejidad temporal como espacial.
    			Son perfectas para organizar, almacenar y distribuir grandes cantidades de información.
                \column{2in}
               Entre ellas, se encuentran: Colas,Pilas,Listas,Arboles,etc.
            \end{columns}
		\end{frame}
	\subsection{Cola,Pilas}	
	    \begin{frame}{Cola,Pilas}
			\justifying
			Estas estructuras permiten almacenar la informacion introducida y darle cierta organizacion, de acuerdo a una prioridad que se le programe. Las pilas son del tipo \textcolor{red}{LIFO}, mientras que las colas son del tipo \textcolor{red}{FIFO}.
		\end{frame}
    \subsection{Arboles}	
	    \begin{frame}{Arboles}
			\justifying
			Ellos se encargan de encontrar datos, cuya distribucion resulte mas compleja; del tipo red neural, grafos, etc.Y da una organización más fácil de visualizar.
			Es muy comun el uso de axiomas, proposiciones, teoremas, corolarios y demostraciones,en este tipo de casos, tal como:
            \begin{block}{Teorema de Pitágoras}
                Sean $a$ y $b$ los catetos de un triángulo rectángulo; y c, la hipotenusa. Entonces, se satisface que
                $c^2=a^2+b^2$
            \end{block}
    	\end{frame}
    \section{Fin}
		\begin{frame}[containsverbatim]{Fin}
              Este es el final de la presentación. Esta parte debería ir con verbatim.
        \end{frame}
\end{document}
