\documentclass[12pt]{article}
\usepackage[utf8]{inputenc}
\usepackage[spanish]{babel}
\usepackage[margin=2cm]{geometry}
\usepackage{multirow}
\usepackage{array}
\usepackage{float}
\title{\textbf{Un Ejemplo con Excel2LaTeX}}
\author{Veronica Jackeline Quiros Díaz   TL41\_VJQD}
\date{July 2020}
\begin{document}
\maketitle
El Cuadro \ref{tab:cotas} es un pequeño ejemplo de lo que puede hacerse usando Excel2LaTeX.
\begin{table}[h]
\begin{tabular}{|l|l|l|l|l|}
\hline\hline
\textbf{$\delta$}& \textbf{Cota de Gilbert-Varshamov} & \textbf{Cota de Hamming} & \textbf{Cota de Elias} &                    \\\hline
0.04     & 0.88180                   & 0.93414         & 0.93356       & 0.0202041028867288 \\\hline
0.08     & 0.79108                   & 0.88180         & 0.87975       & 0.0408336953374561 \\\hline
0.12     & 0.71125                   & 0.83470         & 0.83039       & 0.0619168480353141 \\\hline
0.16     & 0.63884                   & 0.79108         & 0.78376       & 0.0834848610088320 \\\hline
0.20     & 0.57220                   & 0.75009         & 0.73907       & 0.1055728090000840 \\\hline
0.24     & 0.51040                   & 0.71125         & 0.69583       & 0.1282202112918650 \\\hline
0.28     & 0.45283                   & 0.67424         & 0.65375       & 0.1514718625761430 \\\hline
0.32     & 0.39909                   & 0.63884         & 0.61260       & 0.1753788748764680 \\\hline
0.36     & 0.34889                   & 0.60487         & 0.57220       & 0.2000000000000000 \\\hline
0.40     & 0.30204                   & 0.57220         & 0.53243       & 0.2254033307585170 \\\hline
0.44     & 0.25839                   & 0.54074         & 0.49319       & 0.2516685226452120 \\\hline
0.48     & 0.21787                   & 0.51040         & 0.45437       & 0.2788897449072020 \\\hline
0.52     & 0.18044                   & 0.48111         & 0.41592       & 0.3071796769724490 \\\hline
0.56     & 0.14610                   & 0.45283         & 0.37774       & 0.3366750419289200 \\\hline
0.60     & 0.11488                   & 0.42550         & 0.33980       & 0.3675444679663240 \\\hline
0.64     & 0.08687                   & 0.39909         & 0.30204       & 0.4000000000000000 \\\hline
0.68     & 0.06221                   & 0.37356         & 0.26440       & 0.4343145750507620 \\\hline
0.72     & 0.04108                   & 0.34889         & 0.22687       & 0.4708497377870820 \\\hline
0.76     & 0.02379                   & 0.32505         & 0.18942       & 0.5101020514433640 \\\hline
0.80     & 0.01073                   & 0.30204         & 0.15206       & 0.5527864045000420 \\\hline
0.84     & 0.00250                   & 0.27982         & 0.11488       & 0.6000000000000000 \\\hline
0.88     & 0.00006                   & 0.25839         & 0.07811       & 0.6535898384862250 \\\hline
0.92     & 0.00502                   & 0.23775         & 0.04246       & 0.7171572875253810 \\\hline
0.96     & 0.02088                   & 0.21787         & 0.01073       & 0.8000000000000000 \\
\hline\hline
\end{tabular}
\caption{Mi primera tabla usando Excel2LaTeX.}
\label{tab:cotas}
\end{table}
\end{document}
